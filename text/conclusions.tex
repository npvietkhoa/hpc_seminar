\section{Conclusions}

This paper presents and evaluates Petal, a compiler analysis tool designed to convert MPI blocking operations into their corresponding nonblocking versions. 
Petal's main objective is to identify instances where communication can be overlapped with independent work, thereby harnessing the benefits of nonblocking poin-to-point and collective routines introduced in MPI-3, ultimately enhancing application performance. 
To achieve this, Petal replaces blocking calls with their non-blocking counterparts and employs pointer and alias analysis to strategically insert \texttt{MPI\_Wait} statements in the appropriate locations. 
This allows for the overlapping of independent communications and/or computation tasks. 

The utilization of nonblocking collectives in certain cases has been shown to yield benefits for applications. 
This is particularly evident when dealing with sizable messages and when there is a sufficient amount of work available to overlap with communication tasks. 
Generally speaking, there were specific scenarios where the overlapping of communication and computation was feasible. 
These instances of value are expected to be further amplified in the future, especially for applications that have a higher percentage of work dedicated to communication and operate on systems with MPIs that offer robust progress and blocking communication completion (the optimal conditions for overlapping communication and computation).
Additionally, networks with increased message-passing concurrency will also reap the rewards of code restructuring.
